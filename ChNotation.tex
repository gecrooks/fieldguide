
% !TEX encoding = UTF-8 Unicode 
% !TEX root = FieldGuide.tex

\subpart{Appendix}

\Sec{Notation and Nomenclature}
\label{sec:notation}
\index{shape parameter}
\index{scale parameter}
\index{location parameter}
\index{Weibull transform}

\SSec{Notation}
We write $\opr{Amoroso}(x\given a, \theta, \alpha, \beta)$ for a density function, $\oprr{AmorosoCDF}{Amoroso}(x\given a, \theta, \alpha, \beta)$ for the cumulative distribution function, $\opr{Amoroso}(a, \theta, \alpha, \beta)$ for the corresponding random variable, and $X\sim\opr{Amoroso}(a, \theta, \alpha, \beta)$ to indicate that two random variables have the same probability distribution~\cite{Gelman2004}. The semicolon, which we verbalize as ``given'' or ``parameterized by'', separates the arguments from the parameters. \index{given}

\begin{center}
\begin{tabular}{ccll}
parameter  & type &  notes\\
\hline
$a$			&location& power-function \\
$b$			&location& arcsine, $b= a+s$\\
$\pLoc$		 & location &	exponential & eta\\
$\mu$		 & location & normal & mu\\
$\nu$  & location & gamma-exponential & nu\\
$\pLoc$  & location & beta-exponential & zeta\\
$s$			&scale& power function\\
$\pScale$ 		 & scale & 		exponential & lambda\\
$\sigma$ 		 & scale	& normal & sigma\\
$\vartheta^\dagger$ & scale&	log-normal & theta\\
$\theta$ 		& scale & Amoroso & theta\\
$\omega$		 &scale& gen. Fisher Tippett & omega\\
%$c$			& &scale & Levy, stable \\
$\beta$		 & power & power function	& beta \\
$\alpha$ 		& shape	& $>0$, beta and beta prime families & alpha\\
$\gamma$ 	& shape	& $>0$, beta and beta prime families & gamma\\
$n$		 	 &  shape& integer $>0$, number of samples or events \hspace{-2em} \\
$k$	 & shape	& integer $>0$, degrees of freedom\\%, e.g.\ Chi, Students-t, F\\
$m$    		& shape & $>\tfrac{1}{2}$, Pearson IV\\
$v$			& shape & $>0$, Pearson IV\\ % Perhaps use u instead of v, too close to nu\\
\\
&& \footnotesize{$\dagger$ A curly theta, or ``vartheta''.}
\end{tabular}
\end{center}


Throughout I have endeavored to use consistent parameterization, both within families, and between subfamilies and superfamilies. For instance, $\beta$ is always the Weibull power parameter. Location (or translation) parameters: $a$, $b$, $\nu$, $\mu$. Scale parameters: $s$, $\theta$, $\sigma$. Shape parameters: $\alpha$, $\gamma$, $m$, $v$. All parameters are real and the shape parameters $\alpha$, $\gamma$ and $m$ are positive. The negation of a standard parameter is indicated by a bar, e.g.\ $\beta = -\bar{\beta}$. 
For clarity we use a dot `.' in tables of special cases to indicate repetition of the base distribution's parameters. 



\SSec{Nomenclature}
 
 
\paragraph*{interesting} \index{interesting} Informally, an ``interesting distribution'' is one that has acquired a name, which generally indicates that the distribution is the solution to one or more interesting problems.

\paragraph*{generalized-X}\index{generalized}  The only consistent meaning is that distribution ``X'' is a special case of the distribution ``generalized-X''.  In practice, often means ``add another parameter''.  We use alternative nomenclature whenever practical, and generally reserve ``generalized'' for the power (Weibull) transformed distribution. 

\paragraph*{standard-X} \index{standard} \index{standardized} The distribution ``X'' with the location parameter set to 0 and scale to 1. Not to be confused with {\it standardized} which generally indicates zero mean and unit variance. 

\paragraph*{shifted-X} \index{shifted} (or translated-X) A distribution with an additional location parameter. \index{location parameter}

\paragraph*{anchored-X} \index{anchored}\index{ballasted} (or ballasted-X) A distribution with a fixed location (typically with a lower bound set to zero).

\paragraph*{scaled-X} \index{scaled}(or scale-X) A distribution with an additional scale parameter. \index{scale parameter}

\paragraph*{inverse-X}\index{inverse}\index{reciprocal}\index{inverted} (Occasionally inverted-X, reciprocal-X, or negative-X) Generally labels the transformed distribution with $x\to\tfrac{1}{x}$, or more generally the distribution with the Weibull shape parameter negated, $\beta\to-\beta$. An exception is the inverse Gaussian distribution \eqref{InvGaussian}~\cite{Johnson1994}. 

\paragraph*{log-X}  \index{log transform}\index{anti-log transform} Either the anti-logarithmic or logarithmic transform of the random variable X, i.e.\ either  $\exp - \op{X}() \sim \op{log-X}()$ (e.g.\ log-normal)  or $-\ln \text{X}() \sim \op{log-X}()$.  This ambiguity arises because although the second convention may seem more logical, the log-normal convention has historical precedence. Herein, we follow the log-normal convention. 
\label{log-transform-name}

\paragraph*{X-exponential} \index{log transform} The logarithmic transform of distribution X, i.e.\ $-\ln \text{X}() \sim \text{X-exponential}()$. This naming convention, which arises from the beta-exponential distribution \eqref{BetaExp}, sidesteps the confusion surrounding the log-X naming convention.


\paragraph*{reversed-X}\index{reversed} (Occasionally negative-X)  The scale is negated. 
\paragraph*{X of the Nth kind} See ``X type N''.

\paragraph*{folded-X}\index{folded} The distribution of the absolute value of random variable $X$. % See also folded-normal~\eqref{FoldedNormal}

\paragraph*{beta-X} A distribution formed by inserting the cumulative distribution function of X into the CDF of the standard beta distribution \eqref{StdBeta}. Distributions of this form arise naturally in the study of order statistics \secref{OrderStatistic}.

\paragraph*{central-X} A distribution formed by inserting the cumulative distribution function of X into the CDF of the central beta distribution \eqref{CentralBeta}. Distributions of this form arise naturally in the study of median statistics \secref{MedianStatistic}.\cite{\self}


\clearpage
%\twocolumn
\Sec{Properties of Distributions}
\label{PropertiesSec}
\index{MGF|see{moment generating function}}
\index{CGF|see{cumulant generating function}}
\index{CF|see{characteristic function}}


\paragraph*{notation} The multi-letter, camel-cased function name, arguments and parameters used for the probability
density of the family in this text. 



\paragraph*{probability density function (PDF)} 
\index{PDF|see{probability density function}}
\index{probability density function }
\index{density}
The probability density $f_X(x)$ of a continuous random variable is the relative likelihood that the random variable will occur at a particular point. The probability to occur  within a particular interval is given by the integral
\[
P[ a\leq X \leq b] = \int_a^b f_X(x) dx \ . \checked
\notag
\]



\paragraph*{cumulative density function (CDF)} 
\index{CDF|see{cumulative distribution function}}
\index{CCDF|see{complementary cumulative distribution function}}
\index{$F(x)$|see{cumulative distribution function}}
\index{cumulative distribution function}
\index{complementary cumulative distribution function}
\index{distribution function|see{cumulative distribution function}}
The probability that a random variable has a value equal or less than $x$, typically denoted by $F_X(x)$, and also called the distribution function for short.
\[
F_X(x) = \int^x_{-\infty} f_X(z) dz \checked
\notag
\]
The probability density is equal to the derivative of the distribution function, assuming that the distribution function is continuous.
\[
f_X(x) = \frac{d}{dx} F_X(x) \checked
\notag
\]
Negating a scale parameter gives a reversed distribution with the cumulative distribution function replaced by the complementary cumulative distribution function ($\text{CCDF}=1-\text{CDF}$).



\paragraph*{complementary cumulative density function (CCDF)}  (survival function, reliability function)
\index{CCDF|see{complementary cumulative distribution function}}
\index{complementary cumulative distribution function}
\index{survival function}
\index{reliability function}
One minus the cumulative distribution function,  $1-F_X(x)$.\checked The probability that a random variable has a value greater than~$x$. In lifetime analysis the complementary cumulative distribution function is also called the survival function or reliability function.



\paragraph*{support}\index{range}\index{support}\index{image}
The support of a probability density function are the set of values that have non-zero density. The compliment of the support has zero probability. The range (or image) of a random variable (the set of values that can be generated) is the support of the corresponding probability density.



\paragraph*{mode}\index{mode}\index{anti-mode}
The point where the distribution reaches its maximum value. An anti-mode is the point where the distribution reaches its minimum value. 
A distribution is called unimodal\index{unimodal} if there is only one local extremum away from the boundaries of the distribution. In other words, the distribution can have one mode $\frown$ or one anti-mode $\smile$, or be monotonically increasing $/$ or decreasing~$\backslash$.




\paragraph*{mean}\index{mean} The expectation value of the random variable. \index{$\expect$|see{mean}}
\[
\expect[X] = \int x\ f_X(x)\ dx \checked
\notag
\]
Not all interesting distributions have finite means, notably the Cauchy family~\eqref{Cauchy}. Often denoted by the symbol $\mu$.


\paragraph*{variance}\index{variance}\index{standard deviation} The variance measures the spread of a distribution.
\[
\op{var}[X] =  
 \expect\Left[ (X-\expect[X])^2\Right] = \expect\bigl[X^2\bigr] - \expect\bigl[X\bigr]^2
\notag \checked
\]
The variance is also know as the second central moment, or second cumulant, and commonly denoted by the symbol $\sigma^2$. The standard deviation is the square root of the variance.


\paragraph*{central moment}\index{central moment}
\[
\mu_n[X] =  
\expect\Left[ \bigl(X-\expect[X]\bigr)^n\Right]
\checked 
\]
The $n$th moment about the mean. The first central moment is zero, and the second is the variance. 

\paragraph*{skew} \index{skew}  A distribution is skewed if it is not symmetric. A positively skewed distribution tends to have a majority of the probability density above the mean; a negatively skewed distribution tends to have a majority of density below the mean. 

The standard measure of skew is the third cumulant (third central moment) normalized by the $\tfrac{3}{2}$ power of the second cumulant.
\[
\op{skew}[X]    
= \expect\Left[ \Left( \frac{X-\expect[X]} {{\sigma}[X]} \Right)^3 \Right]
= \frac{\kappa_3\phantom{^{\tfrac{3}{2}}}}{{\kappa_2}^{\tfrac{3}{2}}}
\checked
\notag 
\]


\paragraph*{kurtosis} \index{kurtosis}\index{excess kurtosis} Kurtosis measures the spread of a distribution. The normal distribution has zero excess kurtosis. A positive kurtosis distribution longer tails, while a negative kurtosis distribution has shorter tails.

The standard measure of kurtosis is the forth cumulant normalized by the square of the second cumulant.
\[
\op{ExKurtosis}[X] = \frac{\kappa_4}{{\kappa_2}^{2}} \checked
\notag
\]
This measure is called the excess kurtosis to distinguish it from an older definition of kurtosis that used the forth central moment $\mu_4$ instead of the forth cumulant. (Note that  $\tfrac{\kappa_4}{{\kappa_2}^{2}} =  \frac{\mu_4}{{\kappa_2}^{2}} -3$\checked).


\paragraph*{entropy} 
\index{entropy}
The differential (or continuous) entropy of a continuous probability distribution is
\[
\op{entropy}[X] = - \int f(x) \ln f(x)\  dx  \checked
\notag
\]
Note that unlike the entropy of a discrete variable, the differential entropy is not invariant under a change of variables, and can be negative. 

\paragraph*{moment generating function (MGF) } 
\index{moment generating function}
\index{moments}
The expectation
\[
\op{MGF}_X(t) = \expect[e^{t X}]  \ . \checked
\notag
\]
The $n$th derivative of the moment generating function, evaluated at $0$, is equal to the $n$th moment of the distribution. 
\[
\frac{d^n}{dt^n}\op{MGF}_X(t)\Big|_0 = \expect[X^n]  \checked
\notag
\]
If two random variables have identical moment generating functions, then they have identical probability densities.


\paragraph*{cumulant generating function (CGF) }
\label{CGF}
\index{cumulant generating function}
\index{cumulants}
The logarithm of the moment generating function.
\[
\op{CGF}_X(t) = \ln \expect[e^{t X}] \checked
\notag
\]
Note that some authors define the cumulant generating function as the logarithm of the characteristic function.

The $n$th derivative of the cumulant generating function, evaluated at $0$, is equal to the $n$th cumulant of the distribution. 
\[
\frac{d^n}{dt^n}\op{CGF}_X(t)\Big|_0 = \kappa_n(X)  \checked
\]
The $n$th cumulant is a function of the first $n$ moments of the distribution, and the second and third are equal to the second and third central moments.
\begin{align*}
\kappa_1 &= \expect[X] \checked \\
\kappa_2 &= \expect\Left[ (X-\expect[X])^2 \Right] \checked\\
\kappa_3 &= \expect\Left[ (X-\expect[X])^3 \Right] \checked \\
\kappa_4 &= \expect\Left[ (X-\expect[X])^4 \Right] - 3 \expect\Left[ (X-\expect[X])^2 \Right] \checked
\end{align*}
The cumulant expansion, if it exists, either terminates at second order (normal distribution), or continues to infinite order.

Cumulants are often more useful than central moments, since cumulants are additive under summation of independent random variables. 
\[
\op{CGF}_{X+Y}(t) = \op{CGF}_{X}(t) + \op{CGF}_{Y}(t) 
\checked
\notag
\]


\paragraph*{characteristic function (CF)} 
\label{characteristic_function}
\index{characteristic function}
\index{$\phi(t)$|see{characteristic function}}
Neither the moment nor cumulant generating functions need exist for a given distribution. An alternative that always exists is the characteristic function
\[
\phi_X(t) = \expect[e^{i t X}]  \ , \checked
\notag
\]
essentially the Fourier transform of the probability density function. The characteristic function for a  sum of independent random variables is the product of the respective characteristic functions. 
\[
\phi_{ X+ Y}(t) = \phi_{X}( t)\  \phi_{Y}(t) 
\checked
\notag
\]
More generally, the characteristic function of any linear sum of independent random variables is
\[
\phi_{Z}(t) = \prod_i \phi_{X_i}(c_i t), \quad Z = \sum_i c_i X_i \ .
\checked
\notag
\]


\paragraph*{quantile function} 
 \index{quantile function}\index{inverse cumulative distribution function|see{quantile function}}\index{median}
\index{$F$@$F^{-1}(p)$|see{quantile function}}
The inverse of the cumulative distribution function, typically denoted  $F^{-1}(p)$ (or occasionally
 $Q(p)$).
 The median is the middle value of the inverse cumulative distribution function. 
\[
\op{median}[X] = F_X^{-1}(\tfrac{1}{2}) \checked
\notag
\]
Half the probability density is above the median, half below. 
The quantile and median rarely have simple forms.


\paragraph*{hazard function}\index{hazard function}\index{survival function}
The ratio of the probability density function to the complementary cumulative distribution function
\[
\op{hazard}_X(x) = \frac{f_X(x)}{1-F_X(x)} \checked
\notag
\]


